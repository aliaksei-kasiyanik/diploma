\documentclass[a4paper,12pt]{report}

\usepackage{bsustyle/style/bsumain}
\usepackage{bsustyle/style/bsudiplomatitle}
\usepackage{blindtext}
\subfaculty{Кафедра вычислительной математики}


\title{Ускорение сходимости процессов установления. Переобуславливание и подавление компанент}
\author{Касияник Алексей Леонидович}
\mentor{Фалейчик Борис Викторович \\
        кандидат физ.-мат. наук, \\
        доцент кафедры выч. мат.}
\reviewer{Мандрик Павел Алексеевич \\
          заведующий кафедры выч. мат.}

\begin{document}
  \maketitle

  {
    \renewcommand{\contentsname}{Содержание}
    \tableofcontents
  }

  \chapter*{Введение}
  Жесткие задачи исследуются примерно со второй половины 20 века. Однако и сейчас сформулировать точное определение жесткости проблематично. Наиболее прагматическая точка зрения вместе с тем была и исторически наиболее ранней (Кертисс и Хиршфельдер, 1952 год): \textit{жесткие уравнения — это уравнения, для которых определенные неявные методы дают лучший результат, обычно несравненно более хороший, чем явные методы.} При этом определенную роль играют собственные значения матрицы Якоби, но важны и такие параметры, как размерность системы, гладкость решения или интервал интегрирования. 
Более полным является определение данное Ламбертом: \textit{если численный метод с ограниченной областью абсолютной устойчивости, примененный к системе с произвольными начальными условиями вынужден использовать на некотором интервале интегрирования величину шага, которая чрезмерно мала по отношению к гладкости точного решения на этом интервале, тогда говорят, что система является жесткой на этом интервале}.


Как известно, наиболее трудоёмким этапом численного интегрирования жёсткой системы (не)линейных обыкновенных дифференциальных уравнений (ОДУ) размерности n неявным методом является решение на каждом шаге системы (не)линейных уравнений, размерность которой пропорциональна n. В таком случае использование методов ньютоновского типа практически невозможно, а традиционные методы типа простой итерации либо не сходятся, либо сходятся очень медленно. В данной работе рассматриваются способы ускорения методов, основанных на процессах установления, которые применимы в указаной выше ситуации.
В работе исследованы два эффективных способа ускорения сходимости решения в процессе решения методами установления: переобусловливание и подавление компонент. Переобусловливание является классическим способом уменьшения «спектрального числа обусловленности» матрицы решаемой задачи, которое существенным образом определяет свойства сходимости итерационного процесса. С помощью операции переобусловливания производится уменьшение числа обусловленности, что положительно влияет на скорость сходимости процесса.


Также было замечено, что компоненты ошибки, которые соответствуют малым собственным значениям, сходятся медленно. Поэтому предположительно подавление ошибки медленно сходящися компонент может дать существенную прибавку в скорости сходимости итерационного процесса. В результате исследования данного феномена был предложен прием ускорения итерационного процесса, который  по аналогу со схожими алгоритмами из известных источников именуется «подавлением компонент».
Исследованию описанных проблем, разработке вычислительного алгоритмов, которые бы основывались на идее установления и при этом превосходили в скорости известные алгоритмы и посвящается данная работа.
  \label{c:intro}


  \chapter{Жесткие задачи}
  \label{c:stiff_problems}
  Тема численного решения однородных дифференциальных уравнений не нова, и, возможно, несколько удивителен тот факт, что методы, разработанные еще в начале 20 века, до сих пор являются основой наиболее эффективных и распространенных подходов при решении ОДУ. За прошедшее время были достигнуты значительные продвижения в надежности и эффективности этих методов, и большинство существующих типичных научных задач могут быть решены достаточно легко и быстро. Тем не менее есть некоторый класс задач, с которыми классические методы справиться не могут. Такие задачи, называемые «жесткими», слишком важны, чтобы их игнорировать, и слишком трудны, чтобы их решить. Они слишком важны, чтобы их игнорировать, так как они возникают при решении важных физических задач. Они слишком затратные при решении, так как из-за присущей им большой размерности и сложности, классические методы становятся слабо применимы даже несмотря на многократное увеличение мощности современной вычислительной техники. Классические методы решения требуют так много шагов, что ошибки округления могут сделать полученное решение далеким от приемлемого[2]. В этом разделе мы и рассмотрим понятие «жесткости», а также те ключевые проблемы, которые возникают при их решении.

  \section{Явление жесткости}
  \label{s:stiffness}
  Задачи, называемые жесткими, весьма разнообразны, и дать математически строгое определение жесткости непросто. Поэтому в литературе можно встретить различные определения жесткости, отличающиеся степенью строгости. Сущность же явления жесткости состоит в том, что решение, которое необходимо вычислить, меняется медленно, однако в любой его окрестности существуют быстро затухающие возмущения. Характерное время затухания их называют пограничным слоем. Наличие таких возмущений затрудняет получение медленно меняющегося решения численным способом. При этом жесткими могут быть как скалярные дифференциальные уравнения, так и, что встречается особенно часто, системы обыкновенных дифференциальных уравнений.
  
  
\begin{Definition}Система обыкновенных дифференциальных уравнений вида
  \begin{equation}
  \label{stiff:eq}
    u'(t)=Au(t)
  \end{equation}
  с постоянной $(n \times n)$-матрицей $ A $ называется жесткой, если: 
  \begin{enumerate}
  \item $Re\lambda _k <0, k=\overbar{1,n}$ (т.е. задача устойчива);
  \item Отношение $S = \dfrac{\max\limits_{1\leq k\leq n} |Re\lambda _k|}{\min\limits_{1\leq k\leq n} |Re\lambda _k|} $ велико (например, $S > 10$);
  \item Промежуток интегрирования велик по сравнению с длиной погранслоя.[Repn]
  \end{enumerate}
\end{Definition}

Число $S$ иногда называют \textit{коэффициентом жесткости} системы. 


Поскольку система нелинейных обыкновенных дифференциальных уравнений вида $u'(t)=f(t, u(t))$ может быть в окрестности некоторого известного решения $v(t)$ заменена линейной системой 
$$u'(t) = f_u (t, v + \theta (u-v))u + b(t), $$
где $f_u$ - матрица Якоби, а $b(t) = f(t, v)-f_u(t,v + \theta (u-v))v$, то понятие жесткости для нелинейных систем может быть определено аналогично. Заметим, однако, что за пределами класса систем линейных обыкновенных дифференциальных уравнений с постоянной матрицей полагаться на спектр как на источник надежной информации о распространении погрешности уже нельзя[Dekker, Verver].

  
  \section{Ограничение на шаг при решении жестких задач}
  \label{s:stiff_troubles}
  Рассмотрим в качестве примера систему из двух независимых уравнений   
  \begin{equation}
  \label{stiff:system}
  \begin{cases}
   u_1'(t)=-\lambda_1 u_1(t), 
   \\
    u_2'(t)=-\lambda_2 u_2(t), t>0, \lambda_2 \gg \lambda_1 > 0.
  \end{cases}
  \end{equation}
  Эта система имеет решение $u(t)=(u_1(t), u_2(t))^T = (u_1^0 e^{-\lambda_1 t}, u_2^0 e^{-\lambda_2 t})^T$. При выписанных условия на $\lambda_1$ и $\lambda_2$, очевидно, компонента $u_2(t)$ решения затухает гораздо быстрее, чем $u_1(t)$ и, начиная с некоторого момента $t$ поведение вектора $u(t)$ почти полностью определяется компонентой $u_1(t)$. Однако при решении системы \eqref{stiff:system} численным методом величина шага интегрирования, как правило, определяется компонентой $u_2(t)$, не существенной с точки зрения поведения решения системы. Например, используя явный метод Эйлера, мы из первого уравнения имеем ограничения на шаг $\tau \le 2/\lambda_1$, а из второго - $\tau \le 2/\lambda_2$ и, таким образом, ясно, что для решения системы \eqref{stiff:system} как цельного математического объекта шаг $\tau$ ограничен величиной $2/\lambda_2$. 
  
  Такая же ситуация типична и при решении любой системы обыкновенных дифференциальных уравнений вида \eqref{stiff:eq}.
  
Учитывая выше сказанное, можно сделать вывод, что для решения жестких задач наиболее пригодны те численные методы, которые требуют наиболее слабых ограничений на величину шага численного интегрирования из соображений устойчивости. В настояще время наиболее часто для этих целей используют либо неявные методы, либо методы, специально сконструированные для решения задач конкретного вида[Repnikov]. Хорошо применимыми при решении жестких систем являются методы, основанные на процессах установления, которые и рассматриваются в главе~\ref{c:stead_methods}. 
  
  \section{Примеры жестких задач}
  \label{s:stiff_examples}   
  

  \chapter{Методы установления}
  \label{c:stead_methods}
  
  \section{Линейная задача}
  \label{s:linear_problem}
  
  \section{Нелинейная задача}
  \label{s:non_linear_problem} 
  
  \section{Спектральные свойства и скорость сходимости итерационного процесса}
  \label{s:spectral_speed} 
  
  
  \chapter{Ускорение сходимости}
  \label{c:converg_accel}
  
  \section{Переобусловливание}
  \label{s:precond}
    
  \subsection{Первый способ}
  \label{ss:precond_1}
  
  \subsection{Второй способ}
  \label{ss:precond_2}
  
  \section{Подавление компанент}
  \label{s:opress}
  
  \chapter{Численный эксперимент}
  \label{c:numer_ex}
  
  \section{Тестовая задача}
  \label{s:test_problem}
  
  \section{Результаты численного эксперимента}
  \label{s:results}
  
  

  \appendix
\end{document}
